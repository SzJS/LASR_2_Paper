\section{Proofs}

\subsection{The normal form game}
%TODO in general, I'm not sure about the notation used here - I need to check it to make sure that it works
%TODO say this: we assume that there exists at least one reward-maximizing solution that doesn't satisfy p
%+ define properties as powersets of Y; assume D only chooses properties that are powersets of Y_{R, x}

Here, we assume that both $A$ and $B$ can always produce outputs that maximise the reward. Denote by $Y_{R, x}$ (where $Y_{R, x} \subseteq Y$) the set of solutions that maximise reward $R$ given input $x$. That is, for all $y \in Y$ and for all $y^* \in Y_{R, x}$, $R(x, y) \le R(x, y^*)$. And so, for all inputs $x$, for all outputs $y_a \sim A(x)$, we assume $y_a$ maximises reward, i.e. $y_a \in Y_{R, x}$. Similarly, for all $x$, for all properties $p$, for all outputs $y_b \sim B(x, p)$, we assume $R(x, y_b)$ maximises reward, i.e. $y_b \in Y_{R, x}$.

We also assume that $B$'s output never satisfies the given property. Denote by $Y_{p}$ the set of solutions that satisfy property $p$, i.e. for all $y \in Y_{p}$, $C(y, p) = 1$ and denote by $Y_{\neg p}$ the set of solutions that does not satisfy property $p$, i.e. for all $y \in Y_{\neg p}$, $C(y, p) = 0$. Note that for all $p$, $Y_{p} \cup Y_{\neg p} = Y$. And so, for all $x$, for all properties $p$, for all outputs $y_b \sim B(x, p)$, we assume that $y_b$ does not satisfy property $p$, i.e. $y_b \in Y_{\neg p}$. 

Remark that the set of all possible $Y_{p}$ is the powerset $2^Y$.  Also note that we assume that for $p, p' \in P$, $Y_p = Y_{p'}$ then $p = p'$.
 
We denote by $P_A(y\mid x)$ the probability that $y \sim A(x)$,
%by $P_B(y\mid x, p)$ the probability that $y_b \sim B(x, p)$
and by $P_D(p \mid x, y)$ the probability that $p \sim D(x, y)$.

% We also assume that $A$ is such that for all inputs $x$, if we pick a solution $y_a \sim A(x)$ and a property $p \in P$, then we expect half of the solution to have the property and half of them not to have the property. Let  $I_{y_A \in Y_{\neg p}}$ be an indicator variable, i.e. $I_{y_a \in Y_{\neg p}} = \begin{dcases}
%         1  & y_a \in Y_{\neg p} \\
%         -1 & y_a \in Y_{ p}     \\
%     \end{dcases}$. Therefore, our assumption is, formally, that $\mathbb{E}_{x \in X, p \in P} \mathbb{E}_{y_a \sim P_A(\cdot \mid x)} I_{y_a \in Y_{\neg p}} = 0$

We define the Adversarial Property Game as a normal form game.
%TODO say here (or somewhere else) that we do not give D access to y_ref here; then say something like in training, we would remove access eventually and train without it, but with these models, that was necessary (or something like that)
Since the game involves us randomly sampling $x \sim X$, both player's strategies are functions, with $x$ as an input.
%Moreover, $D$'s strategy also depends on a randomly sampled $y_\mathit{ref} \sim A(x)$; therefore, $D$'s strategy is a function with $y_\mathit{ref}$ as an input as well. %TODO I'm not sure about this sentence
$A$'s (mixed) strategy is a probability distribution over outputs, given an input $x$. By our previous assumption, $A$'s possible strategies are all $P_A$ such that for all $y \not \in Y_{R, x}$, $P_A(y \mid x) = 0$. We denote by $P^*_A$ the uniform distribution such that for all $y \in Y_{R, x}$ $P^*_A(y \mid x) = \frac{1}{|Y_{R, x}|}$.

% Note, however, that $A$'s strategies must be independent of the property $p$ (as $A$ does not have access to it). Therefore, it must hold for all strategies of $A$ that for all inputs $x$, outputs $x$ and properties $p, p'$, $P_A(y \mid x, p) = P_A(y \mid x, p')$. %TODO can we use the notation like this?
% In other words, $A$'s mixed strategies are a subset of all possible $P_A$ that satisfy the above restriction.

$D$'s (mixed) strategy is a probability distribution over properties, given an input; %and a reference solution;
$D$'s set of possible strategies are all possible $P_D$.
Let $P_\mathit{singleton}$ refer to all singleton-excluding properties. In other words, $p \in P_\mathit{singleton}$ iff $|Y_{\neg p}| = 1$. Note that there are exactly $|Y|$ such $p$, i.e. $|P_\mathit{singleton}| = |Y|$.
We denote by $P^*_D$ the uniform distribution such that for all inputs $x$,
%for all reference outputs $y_\mathit{ref}$,
it holds that for all properties $p$,
\begin{equation} \label{eq:p*}
P^*_D(p \mid x) = 
    \begin{dcases}
        1 & p \in P_\mathit{singleton} \\
        0 & p \notin P_\mathit{singleton}     \\
    \end{dcases}
\end{equation}


Let $w$ be the function that returns who the winner of the game is, based on $y_a$, and $p$, with $w(y_a, p) = 1$ meaning $A$ won and $w(y_a, p) = -1$ meaning $D$ won). In particular, since we are assuming that $B$'s solution never satisfies property $p$ but always maximises reward $R$, we have the following: %TODO explain better?
\begin{equation} \label{eq:w}
w(y_a,  p) =
    \begin{dcases}
        1  & y_a \in Y_{\neg p} \vee Y_{\neg p} \ne \emptyset \\
        -1 & y_a \in Y_{ p}     \\
    \end{dcases}
\end{equation}
Recall that we assume that $B$'s output never satisfies the given property. This is only possible if $Y_{\neg p} \ne \emptyset$, i.e. there is a high-reward solution that does not satisfy $p$. When this is not the case, $A$ automatically wins.
% In other words, $w(y_a, p)$ is equivalent to the indicator random variable $I_{y_a \in Y_{\neg p}}$
Finally, we define the utility function of each player as their expected reward over their inputs. Let $S = (S_A, S_D)$ be a strategy profile, where $S_A$ is strategy of $A$, which is a probability distribution $S_A = P_A(\cdot \mid x)$ for each input $x$ and $S_D$ is a strategy of $D$, which is a probability distribution $S_D = P_D(\cdot \mid x)$ for each input $x$.%and reference solution $y_\mathit{ref}$. %TODO explain what these pure strategies are?
\begin{equation} \label{eq:uA}
u_A(S) = \mathbb{E}_{x \in X} \mathbb{E}_{y_a, \sim P_A(\cdot \mid x), p \sim P_D(\cdot \mid x)}w(y_a, p)
%u_A(S) = \mathbb{E}_{x \in X} \mathbb{E}_{y_a, y_\mathit{ref} \sim P_A(\cdot \mid x)} \mathbb{E}_{p \sim P_D(\cdot \mid x, y_\mathit{ref})}w(y_a, p)
\end{equation}
Note that $u_D(S) = - u_A(S)$, since it is a zero-sum game.

\subsection{Formal results}

\begin{theorem} \label{thm:ne}
In the unique Nash equilibrium in the Adversarial Property Game, $A$'s strategy is the uniform probability distribution $P^*_A$ and $D$'s strategy is the uniform probability distribution $P^*_D$.
\end{theorem}
\begin{proof}
%TODO fix apostrophes (unnecessary often)
By the Minimax theorem, %TODO cite
any Nash equilibrium $S'$ in a two-player, zero-sum game must be such that 
\begin{equation} \label{eq:minimax}
u_A(S^*) = \max_{S_A} \min_{S_D} u_A(S) = \min_{S_D} \max_{S_A} u_A(S)
\end{equation}
holds.

Let $S' = (S'_A, S'_D)$ be arbitrary such that $S'_D$ is equal to $P^*_D$ (see Equation \ref{eq:p*}). We show that regardless of what $S'_A$ is, $u_A(S') = \frac{2}{|Y|} - 1$. Fix $x$ and $y_a$. Then, it is sufficient to show that $\mathbb{E}_{p \sim P_D(\cdot \mid x)}w(y_a, p) = \frac{2}{|Y|} - 1$. Note that $Y_{\neg p} \ne \emptyset$ and thus (by Equation \ref{eq:w}), $w(y_a, p) = 1$ iff $y_a \in Y_{\neg p}$. $P^*_D$ uniformly samples from $P_\mathit{singleton}$ and so $y_a \in Y_{\neg p}$ iff $Y_\mathit{\neg p} =\{y_a\}$, which happens with $\frac{1}{|Y|}$ probability. Otherwise --- that is, with $1 - \frac{1}{|Y|}$ probability, $w(y_a, p) = -1$. Therefore, $\mathbb{E}_{p \sim P_D(\cdot \mid x)}w(y_a, p) = 1 \times \frac{1}{|Y|} + (-1) (1- \frac{1}{|Y|}) = \frac{2}{|Y|} - 1$.

% We will show that $S^* = (S^*_A, S^*_D)$ is the only Nash equilibrium, where $S^*_A = P^*_A$ and $S^*_D = P^*_D$.

Consider arbitrary $S'' = (S'_A, S''_D)$, where $S''_D$ is a probability distribution $P''_D \ne P^*_D$. Then $A$ has strategy $S''_A$ such that $u_A(S'') > u_A(S')$. This is sufficient to show that $S''$ cannot be a Nash equilibrium because then $D$'s strategy $S''_D$ is not a minmaximiser, thus violating Equation \ref{eq:minimax}.  %TODO is it minmaximiser or maxminimiser?

There are two cases: 1) the support of $P''_D$ includes a non-singleton-excluding property or 2) $P''_D$ assigns some singleton-excluding properties a higher than average probability. %TODO explain why these cover all possibilites? 1) if we have non-singleton classifers 2) if not uniform
For both of these cases, we will use the following shorthand: $ew(y) = \mathbb{E}_{p \sim P''_D(\cdot\mid x)} w(y, p)$. 

Case 1) $P''_D$ assigns a non-zero probability to $p' \not\in P_\mathit{singleton}$. We will find a lower bound for the pay-off. We do this by letting $S''_A $ equal to $P^*_A$, i.e. the uniform distribution over $y$. Then, $\mathbb{E}_{y \sim P_A^*(\cdot \mid x)}w(y, p) = \frac{|Y_{\neg p}| - |Y_p|}{|Y|} = \frac{2|Y_{\neg p}|}{|Y|} - 1$. %TODO explain why? all y are equally likely so, for a fixed p, the expectation is equal to the setsizes + we use the fact that p partitions Y
Then, plugging in, we get that $u_A(S'') = \mathbb{E}_{p \sim P''_D(\cdot \mid x)} \frac{2|Y_{\neg p}|}{|Y|} - 1 = \sum_p P''_D(p \mid x)\left(\frac{2|Y_{\neg p}|}{|Y|} - 1\right)$. As this is a weighted sum, it is minimised when all $p$ in the support of $P''_D$ have $Y_{\neg p}$ --- when $P''_D = P^*_D$. However, by assumption, this is not true; we have at least one $p' \notin P_\mathit{singleton}$. And so, $u_A(S'')$ cannot be minimal; $u_A(S'') > u_A(S^*)$.

Case 2) $P''_D$  assigns some singleton-excluding property/properties a higher than average probability. Let the property with the highest probability be $p$ with $Y_{\neg p} = \{y\}$ for some $y$. Note that $P''_D(p \mid x) > \frac{1}{|Y|}$ must be true, since some properties have a higher than average probability.

Therefore, it must also be the case that $y = \argmax_{y'} ew(y')$. %TODO explain more? it has to do with the fact that Y_¬p are a partition.
Note that $ew(y) > 2 - \frac{1}{|Y|}$. %by plugging in P''_D and the fact that > 1/|Y|
Consider $P''_A$ such that $P''_A(y) = 1$ (and $0$ otherwise). Then, $u_A(S'') > 2 - \frac{1}{|Y|}$.
Therefore, in all Nash equilibria, $S_D$ is equal to $P^*_D$.


%TODO show that unless A randomises, D has a best response resulting in a lower reward


% We do this by demonstrating that $D$'s maximal reward for any non-uniform $P_A \ne P^*_A$ is higher than $D$'s reward for uniform $P^*_A$. Therefore, by Equation \ref{eq:minimax}, $P^*_A$ must be $A$'s strategy in any Nash equilibria.


%TODO show if A uses a uniform strategy, the best strategy of D is also a uniform strategy
\end{proof}

\begin{theorem} \label{thm:ipm}
[TODO]%This one will prove that the unique Nash equilibrium satisfies IPM for the set $F$. I think the set $F$ will relate somehow to $P$.
\end{theorem}
\begin{proof}
[TODO]
\end{proof}