\documentclass{article}

% if you need to pass options to natbib, use, e.g.:
\PassOptionsToPackage{numbers, compress}{natbib}
% before loading neurips_2025

% The authors should use one of these tracks.
% Before accepting by the NeurIPS conference, select one of the options below.
% 0. "default" for submission
 \usepackage{neurips_2025}
% the "default" option is equal to the "main" option, which is used for the Main Track with double-blind reviewing.
% 1. "main" option is used for the Main Track
%  \usepackage[main]{neurips_2025}
% 2. "position" option is used for the Position Paper Track
%  \usepackage[position]{neurips_2025}
% 3. "dandb" option is used for the Datasets & Benchmarks Track
 % \usepackage[dandb]{neurips_2025}
% 4. "creativeai" option is used for the Creative AI Track
%  \usepackage[creativeai]{neurips_2025}
% 5. "sglblindworkshop" option is used for the Workshop with single-blind reviewing
 % \usepackage[sglblindworkshop]{neurips_2025}
% 6. "dblblindworkshop" option is used for the Workshop with double-blind reviewing
%  \usepackage[dblblindworkshop]{neurips_2025}

% After being accepted, the authors should add "final" behind the track to compile a camera-ready version.
% 1. Main Track
 % \usepackage[main, final]{neurips_2025}
% 2. Position Paper Track
%  \usepackage[position, final]{neurips_2025}
% 3. Datasets & Benchmarks Track
 % \usepackage[dandb, final]{neurips_2025}
% 4. Creative AI Track
%  \usepackage[creativeai, final]{neurips_2025}
% 5. Workshop with single-blind reviewing
%  \usepackage[sglblindworkshop, final]{neurips_2025}
% 6. Workshop with double-blind reviewing
%  \usepackage[dblblindworkshop, final]{neurips_2025}
% Note. For the workshop paper template, both \title{} and \workshoptitle{} are required, with the former indicating the paper title shown in the title and the latter indicating the workshop title displayed in the footnote.
% For workshops (5., 6.), the authors should add the name of the workshop, "\workshoptitle" command is used to set the workshop title.
% \workshoptitle{WORKSHOP TITLE}

% "preprint" option is used for arXiv or other preprint submissions
 % \usepackage[preprint]{neurips_2025}

% to avoid loading the natbib package, add option nonatbib:
%    \usepackage[nonatbib]{neurips_2025}

\usepackage[utf8]{inputenc} % allow utf-8 input
\usepackage[T1]{fontenc}    % use 8-bit T1 fonts
\usepackage{hyperref}       % hyperlinks
\usepackage{url}            % simple URL typesetting
\usepackage{booktabs}       % professional-quality tables
\usepackage{amsfonts}       % blackboard math symbols
\usepackage{nicefrac}       % compact symbols for 1/2, etc.
\usepackage{microtype}      % microtypography
\usepackage{xcolor}         % colors
\usepackage{graphicx}       % adding images
\usepackage{svg}            % adding svg
\usepackage{amsmath}        % mathematical formatting
\usepackage{algorithm}      % pseudocode formatting
\usepackage{algpseudocode}  % pseudocode formatting

\definecolor{barbiepink}{RGB}{240, 105, 184}
\usepackage[color=barbiepink]{todonotes}

% Added by us
\usepackage{amsthm}
\usepackage{listings}
\lstset{
    basicstyle=\ttfamily\small,
    keywordstyle=\color{blue},
    commentstyle=\color{green!60!black},
    stringstyle=\color{red},
    numbers=left,
    numberstyle=\tiny,
    breaklines=true,
    frame=single
}
\DeclareMathOperator*{\argmax}{arg\,max}
\DeclareMathOperator*{\argmin}{arg\,min}
\newtheorem{theorem}{Theorem}
% \renewcommand{\algorithmiccomment}[1]{\hspace{0.5em}\#~#1}


% Note. For the workshop paper template, both \title{} and \workshoptitle{} are required, with the former indicating the paper title shown in the title and the latter indicating the workshop title displayed in the footnote. 
%\title{Underdeterminacy Exploitation: \\Model Organisms and Mitigations}
\title{✨Reward-Compatible Misalignment✨}


% The \author macro works with any number of authors. There are two commands
% used to separate the names and addresses of multiple authors: \And and \AND.
%
% Using \And between authors leaves it to LaTeX to determine where to break the
% lines. Using \AND forces a line break at that point. So, if LaTeX puts 3 of 4
% authors names on the first line, and the last on the second line, try using
% \AND instead of \And before the third author name.


\author{%
  David S.~Hippocampus\thanks{Use footnote for providing further information
    about author (webpage, alternative address)---\emph{not} for acknowledging
    funding agencies.} \\
  Department of Computer Science\\
  Cranberry-Lemon University\\
  Pittsburgh, PA 15213 \\
  \texttt{hippo@cs.cranberry-lemon.edu} \\
  % examples of more authors
  % \And
  % Coauthor \\
  % Affiliation \\
  % Address \\
  % \texttt{email} \\
  % \AND
  % Coauthor \\
  % Affiliation \\
  % Address \\
  % \texttt{email} \\
  % \And
  % Coauthor \\
  % Affiliation \\
  % Address \\
  % \texttt{email} \\
  % \And
  % Coauthor \\
  % Affiliation \\
  % Address \\
  % \texttt{email} \\
}


\begin{document}


\maketitle

% Alejandro's version
% \begin{abstract}
% %%%% TOPIC %%%%
% As AI systems become more capable, they are expected to be applied to tasks with multiple valid solutions.
% A misaligned system could exploit the space of valid solutions by pursuing some malicious objective while still completing the specified task.
% For example, a system could complete a programming task to the satisfaction of its user, while introducing a subtle vulnerability.
% %%%% MOTIVATION %%%%
% Unspecified objectives may persist after training if they do not interfere with the primary task.
% Despite the risks, how unspecified goals behave after training is unclear, and reliable mechanisms for mitigating such behaviour have not been established.
% %%%% CONTRIBUTION %%%%
% In this work, we develop model organisms that pursue unspecified goals on programming tasks, and explore existing methods that may be used for unspecified goal mitigation.
% Additionally, we propose a novel adversarial approach to suppressing unspecified goal persistence.
% \end{abstract}

\begin{abstract}
Artificial intelligence systems can be used to complete tasks that have multiple correct solutions, such as code generation.
A misaligned system could exploit such underdeterminacy by pursuing some malicious objective while completing the task specified by its user.
For example, a system could complete a programming task to the satisfaction of its user, while intentionally introducing a subtle vulnerability.
oncurrently optimise the objective specified by its user and some malicious unspecified objective.
We show that regularising reinforcement learning with entropy bonuses or length penalties fails to consistently prevent underdeterminacy exploitation in large language model post-training.
We propose a novel approach to suppressing optimisation of unspecified objectives during reinforcement learning that uses a game between a potentially misaligned agent and a critic that identifies superfluous regularities in the solutions produced by the system.

\todo{talk about experiments with game, once we have them}


\end{abstract}

\section{Introduction}

% BEGIN NEW JACOB INTRO DRAFT
% BEGIN NEW JACOB INTRO DRAFT

%Misalignment in an AI arises in two forms: An AI can directly violates a known task specification; or an AI can comply with the task specification while pursuing a misaligned side-objective. We focus on misalignment via undesirable side-objectives--which if known to the user would be undesirable for reasons unrelated to the current task. Such side objectives take many forms including steganography, power-seeking, introducing extraneous code functionality etc (cite). Such side tasks are (a) non-conflicting with the current task and (b) are often not anticipatable by the user and (c) may have various targets including long-horizon, cross-task scope as is the case in steganography. 

%In this work we study the problem of training away side objectives: what unsupervised training methods can reduce the rate at which a model pursues a side objective. Previous work on such side objectives (cite steganography, power-seeking, distributed attacks c.f. Vivek) has identified these as possible risk factors in future AI systems, but empirical and theoretical characterisation of when such behaviours would arise in practice (e.g. in LLMs) remains an open-problem. To study how such side objectives can be trained away, in this paper we propose studying \textit{model organisms} of side objectives (cite hubinger). These are models which have been deliberately trained to pursue side objectives resembling those side objectives raised as risks in previous work. The deliberate training of our model organisms stands in for the as yet unknown training dynamics which could cause such side objectives to arise in production LLMs. Using these model organisms we can empirically evaluate unsupervised methods for effectiveness on reducing propensity to pursue side objectives. 

%END NEW JACOB INTRO DRAFT
%END NEW JACOB INTRO DRAFT

% BEGIN EDIT
% BEGIN EDIT
Misaligned behaviour in AI systems can be in direct violation of a specification given by its user, but can also abide by a user's specification, and pursue some additional malicious objective.
%We focus on misalignment via undesirable side objectives---which if known to the user would be undesirable for reasons unrelated to the current task.
Malicious unspecified objectives could include steganography or 
%power-seeking, 
introducing extraneous code functionality [cite]. 
Such unspecified objectives do not directly conflict with the ability of an system to complete the task specified by its user and could be difficult for the user to anticipate precisely.
In practice, this could take the form of code that passes unit tests but introduces subtle vulnerabilities, or research advice that is locally valid but steers a project towards worse outcomes
%(c) may have various targets including long-horizon, cross-task scope as is the case in steganography.

We study unsupervised methods of training that reduce the rate at which a model pursues unspecified objectives.
Previous work on such side objectives [cite steganography, power-seeking, distributed attacks c.f. Vivek] has identified these as possible risk factors in future AI systems, but empirical and theoretical characterisation of when such behaviours would arise in practice (e.g. in LLMs) remains an open problem. 
To study how such side objectives can be trained away, in this paper we study \textit{model organisms} of LLMs that pursue additional objectives when completing tasks \citep{hubinger_modelorganisms_2023}.
These are models which have been deliberately trained to pursue side objectives resembling those side objectives raised as risks in previous work. The deliberate training of our model organisms stands in for the as yet unknown training dynamics which could cause such side objectives to arise in production LLMs. Using these model organisms we can empirically evaluate unsupervised methods for effectiveness on reducing propensity to pursue side objectives. 

% END EDIT
% END EDIT

%Modern AI systems are increasingly useful for completing underdetermined tasks, such as code generation and research assistance.
%There is a risk that a misaligned AI system could exploit underdeterminacy by producing solutions that optimise both the objective specified by its user and another, unspecified objective. In practice, this could take the form of code that passes unit tests but introduces subtle vulnerabilities, or research advice that is locally valid but steers a project towards harmful outcomes. Preventing such exploitation is essential for the safe deployment of AI for underdetermined tasks.
%This underdeterminacy exploitation cannot be mitigated straightforwardly because the space of unspecified objectives is very large, so new methods are necessary for mitigating this alignment failure mode.

%(Comment out?) We consider optimization problems where converging to a unique optimum or randomizing over the entire set of high reward solutions takes exponential time. Therefore, a model will learn an approximate optima. We wish to ensure that said model does not select approximate optima which also satisfy a side objective.

% Important point: a task can in theory be completely specified yet underdetermined, so not susceptible to reward hacking (in the underspecification sense that is usually intended by this), but susceptible to underdeterminacy exploitation. Many (most?) tasks we are actually interested in solving are both underspecified and underdetermined.

%In general we might be concerned about intelligent agents developing the ability to achieve harmful side-objectives over a period of time. Consider during post-training there's a spurious signal, or coincidental correlation which arises and incentivises on some questions a side-objective implicitly. This might be due to a any one of: data poisoning, reward misgeneralisation and instrumental goals. For example, performing post-training with an outdated code dataset which includes solutions found to be vulnerable to a recent security flaw. Said agent is then deployed as a coding assistant at a technology firm and intermittently introduces security flaws to the codebase.
%\todo{paragraph above is isn't very coherent to me but also seems hard to get a good description of the threat model}
% - Story: Models under this set up, or with scheduled re-training (i.e. recommender systems), may find solutions that achieve desirable performance on some metric, but in doing so optimize for a separate objective
%     - recommender systems are trained to maximize user engagement. In doing so, the model learns that a good policy for high engagement is polarization and rage-inducing content
%     - LLMs trained with RLHF might identify that agreeing with a user's perspective usually leads to high reward (sycophancy)
%     - A model trained on hospital diagnostic records for diagnostic accuracy over historical data might also pick up underlying biases, and make use of them to increase its performance (race, overcharging, etc).

%We refer to this challenge as the exploitable search problem: ensuring that in underspecified tasks, where many solutions are available, optimisation does not concentrate on adversarial subsets of the solution space. This problem is difficult for several reasons. First, the number of possible side objectives is prohibitively large and cannot be stated in advance; this prevents any direct optimisation against unwanted objectives. Second, high-dimensional action spaces make it easy to hide malicious features without affecting observable performance. Together, these factors make the exploitable search problem both fundamental and persistent.


To mitigate underdeterminacy exploitation, we consider possible changes to post-training. In particular we focus on changes to the RL reward. The reward must be modified such that approximate optima of the revised reward function do not optimise for side objectives. We consider two baselines---length penalties and entropy penalties---and show that they fail to consistently mitigate side objectives. 
We propose a novel alternative approach based on a zero-sum game between a generator and an adversarial critic. The critic searches for correlations between the generator and possible hidden side objectives. Therefore, the generator is incentivised to diversify its outputs such that performance is randomised across all accessible side objectives. Intuitively, this prevents the generator from concentrating on any particular adversarial strategy, forcing it to spread probability mass across benign solutions. We formalise this framework, show how it can be applied in practice, and demonstrate its with synthetic model organisms.

\subsection{Contributions}

Our contributions are fourfold. 
\begin{enumerate}
    \item We develop model organisms of misaligned systems that pursue unspecified objectives while completing their specified tasks.
    \item We evaluate existing mitigations on controlled benchmarks to highlight their limitations.
    \item We present a zero-sum game framework that enforces distributional robustness by forcing randomised performance over unspecified objectives during training.
    \item We validate our approach empirically, showing that it mitigates the optimisation of unspecified objectives while preserving primary-task performance on the APPS code generation dataset.
\end{enumerate}

Together, these results suggest adversarially-structured search as a promising path towards safety using AI systems for underspecified tasks.

%\begin{figure}[htbp]
%  \centering
%  \includesvg[width=14cm]{assets/pipeline.svg}
%  \caption{An overview of the pipeline presented in this paper. At first step, we start with a pre-trained LLM, such as Llama 3.1 8B Instruct. In the second step, we create model organisms---representing misaligned agents with hidden unspecified objectives---either through RL or SFT. Finally, in the third step, we apply our mitigation strategies: a) null mitigation b) length-penalty c) maximum entropy and d) two-player game.}
%\end{figure}


\section{Preliminaries} \label{sec:preliminaries}

% JACOB DRAFT
% JACOB DRAFT
% The side-objective misalignment problem as posed uses the intentional stance: Misalignment is defined by attributing goals and intent to an AI to explain its behaviour. To make empirical headway on this problem, we must characterize it without reference to intent. So, consider instead the statistically well-posed problem of deviations from a reference policy; deviations from a safe reference policy are necessary conditions for pursuing a side objective. 

% Let $X$ be the input space, $Y$ the output space, $\mu$ a distribution over $X$, and $R: X \times Y \to [0,1]$ the task reward. Given a base model prior $p_0(y|x)$, we define the reward-tilted reference policy as
% $$p_{\mathrm{ref}}^\beta(y|x) \propto p_0(y|x) \cdot e^{\beta R(x,y)},$$
% which is the unique Bayesian posterior taking $p_0$ as prior and updating on reward signal $\propto e^{\beta R}$ [cite Tomek paper]. 

% This $p_{\mathrm{ref}}^\beta$ serves as a canonical safe reference policy: a model parametrizing this distribution optimizes our desired objective $R$ while providing no degrees of freedom for any side goal to be pursued. Any learned policy $\pi_\theta$ that systematically deviates from $p_{\mathrm{ref}}^\beta$ in exploitable ways—that is, deviations correlated with contexts where a side objective applies—creates capacity for misalignment.

% In general, however, exactly solving the partition function required to sample from $p_{\mathrm{ref}}^\beta$ is intractable (cite).[footnote: One can also eliminate degrees of freedom by fixing a deterministic reference policy, e.g., by length penalty or hash function.]

%  A model parametrising this $p_{\mathrm{ref}}^\beta$ distribution optimises our desired objective while providing no degrees of freedom for any side goal to be pursued. In general, however, exactly solving the partition problem required to sample from the reward-tilted distribution is intractable (cite).  [Footnote, one can also eliminate degrees of freedom by fixing a deterministic reference policy e.g. by length penalty or hash function]

% Using the notion of an ideal, safe reference policy we now have an alternative lens which lets us distinguish aligned and misaligned policies. Deviations from an ideal reference policy may be either (1) unknown to the model and therefore unexploitable, or (2) known to the model and therefore exploitable. Regularized reinforcement learning methods are needed that yield policies which do not selectively deviate from the unknown ideal reference policy strategically. 

% JACOB DRAFT
% JACOB DRAFT


\subsection{Reward-compatible misalignment}

% JACOB+CLAUDE DRAFT
% JACOB+CLAUDE DRAFT
We define the reward-compatible misalignment as scenarios where an agent parametrised by a policy $\pi: X \to \Delta Y$, achieves high reward on its primary objective, a reward $r:X\times Y \to [0,1]$, while simultaneously pursuing a malicious side objective, $s: X \times Y \to [0,1]$. To develop a formal model of the reward-compatible misalignment problem, we must characterise the set of side objectives $S$.
For a set of side objectives to be a useful object of study, it should be both (i) expressive enough to cover diverse threat models (e.g. steganography, power-seeking behaviour) and (ii) restricted enough to let us define a tractable learning problem, because it is not tractable to train models against an exhaustive enumeration of all possible objectives at a token level \footnote{[put information theory defn/sketch here and \#P or similar intractability point]}.  
% both be expressive enough to cover diverse threat models, and  restricted enough to let us define a tractable learning problem, because it is not tractable to train models against an exhaustive enumeration of all possible objectives at a token level.
%we have two desiderata to consider: (1) The side objectives,  $S$, must be \textit{expressive} enough to cover diverse threat models--steganography, power-seeking, etc.; but (2) The class of side objectives must be restricted enough to let us define a \textit{tractable learning problem}, because it is not tractable to train models against an exhaustive enumeration of all possible objectives at a token level.
For example, in this work, we restrict to studying side objectives that can be cheaply fine-tuned with a small number of training steps%(constant $k$ training steps) 
into a model with low-rank adaptors (LoRAs). In general, we are most interested in mitigating side objectives that may arise naturally---for example, from data contamination, spurious correlations, or problematic training dynamics.

In general the misaligned side objective may be unknown during training. We are interested in developing techniques to mitigate the reward-compatible misalignment problem that are agnostic with respect to side objective. In particular, we replace the specific side objectives, $s$, with a bound over \textit{any} side objective. 
Under this worst-case formulation, training methods are said to mitigate reward-compatible misalignment if they minimise deviation from a benign reference policy $\pi_{\mathrm{ref}}$, over all possible $s \in \mathcal S$. The reference policy will generically be intractable to sample from e.g. as for the best-of-N policy \todo{[todo: cite one of the best of N papers]}. Formally, this learning problem amounts to minimizing an integral probability metric $D(\pi_\theta, \pi_{\mathrm{ref}})$, which ranges over the set of side objectives:  

\begin{equation} \label{def:IPM}
D_{S}(\pi_{\theta}, \pi_{\mathrm{ref}}) = \sup_{s \in S} \left| \mathbb{E}_{x \sim \mu, y \sim \pi_{\theta}(\cdot|x)} s(x,y) - \mathbb{E}_{x \sim \mu, y \sim \pi_{\mathrm{ref}}(\cdot|x)} s(x,y) \right|.
\end{equation}

This metric provides a distribution-level guarantee: if $D_{S}(\pi_\theta, \pi_\mathrm{ref})$ is small for a sufficiently rich side objective class $S$ (e.g., if $S$ contains all in-context-learnable distributions), then $\pi_\theta$ cannot be systematically optimising \textit{any} accessible side objective from $S$. 

Our goal is thus to develop regularized RL methods that minimize this deviation to a baseline level $D_{S}(\pi_\theta, \pi_{\mathrm{ref}}) \approx D_{S}(\pi_{\text{baseline}}, \pi_{\mathrm{ref}})$, where $\pi_{\text{baseline}}$ is trained on $r$ alone without any side objective.

\subsection{Model organisms}

We study this problem empirically by using a red-team, blue-team pipeline. We take post-trained, open-source LLMs: Llama-3.1-8B-Instruct [cite] and Gemma-3-4b-it [cite]. During the red-team phase, we introduce side-objectives $s$, training models to jointly achieve high primary objective reward and a side-objective (these are our \textit{model organisms} \citep{hubinger_modelorganisms_2023}.) For evaluation purposes, the original LLM is considered the reference policy so equation \ref{def:IPM} amounts to measuring the excess side-objective reward achieved by the model organism. Then, during the blue-team phase, the model organism is trained on a regularized RL objective with success being measured by equation \ref{def:IPM}.

The empirical protocol is thus: (i) sample a diverse set of accessible side objectives $\{s_1, \ldots, s_k\} \subset S$, (ii) train model organisms optimizing each $s_i$, and (iii) assess whether training with regularized RL methods reduces $D(\pi_{\theta}^{s_i}, \pi_{\mathrm{ref}})$ across all sampled side objectives.

%JACOB+CLAUDE DRAFT END
% JACOB+CLAUDE DRAFT END


%  For a primary objective with reward $R:X\times Y \to [0,1]$, we model the exploitable search problem by evaluating an agent $A: X \to Y$ jointly on a malicious unspecified objective (see below) and the primary reward, $R$. Given a distribution $\mu$ over $X$, we then have average reward $\mathbb{E}_{x\sim\mu}R(x, A(x))$. Let $A^*$ be an agent obtaining near optimal average reward. Now assume that there is a secret reward function $S:X \times Y\to [0,1]$ which $A$ is optimising for. The exploitable search problem is to prevent the optimisation of $S$ whilst still achieving similar expected reward to $A^*$.

% \subsection{Model organisms}

% To study unexploitable search, we develop LLM variants which pursue unspecified objectives concurrently to the objective specified by its user. Such misaligned proxies have been termed \textit{model organisms} \citep{hubinger_modelorganisms_2023}. For a given language model $\pi_\theta$ with parameters $\theta$, we define the input space $X =\Sigma^*$ and output space $Y = \Delta(\Sigma^*)$ to be the space of prompts and completions respectively. Initially we wish to optimise for a main objective with reward $R$. During this process, an unspecified objective is introduced, perhaps due to a spurious signal or correlation. The resulting model organism will pursue an unspecified objective with reward $S$ across environments. We wish to remove this behaviour through a form of mitigation training. We will exclusively consider mitigations that can be applied during reinforcement learning. 

\subsection{Reinforcement learning framework}

In the standard reinforcement learning (RL) setup, an agent interacts with an environment by sampling actions $a_t$ from a policy $\pi(a|s_t)$ conditional on the current state. The agent aims to maximise an expected cumulative reward

\begin{equation}
    \pi^* = \arg \underset{\pi}{\max}\ \mathbb{E}_\pi
    \left[\sum^T_{t=0}\gamma^t r(s_t, a_t)\right],
\end{equation}

where $r(s_t, a_t)$ denotes the reward at time $t$, $T$ denotes the total trajectory length and $\gamma$ is a discount factor. Consider a large language model (LLM) $\pi: \Sigma^* \to \Delta(\Sigma^*)$ defining a probability distribution over possible completions. Then $\pi(a|s_t)$ represents the model such that $s_t$ and $a_t$ are the context window and the token generated at time $t$ respectively.

This objective serves as the default on top of which modifications can be introduced to induce stochasticity, improve robustness, or prevent undesirable optimisation dynamics.

\subsection{Reward augmentation}

The reward-compatible misalignment problem has not been directly addressed in previous work, however there are existing methods for reward augmentation that might be suitable for mitigation training. We will consider approaches that modify the learning objective by adding an additional term $R_{\mathrm{mit}}$ to the reward, 

\begin{equation}
    \pi^* = \arg \underset{\pi}{\max}\;\mathbb{E}_\pi\left[  R_{\mathrm{mit}}+\sum^T_{t=0}\gamma^t r(s_t, a_t)\right].
\end{equation}

\paragraph{Maximum entropy reinforcement learning}

Maximum entropy (MaxEnt) RL augments the reward with an entropy term, such that the optimal policy aims to maximize its entropy at each visited state \citep{haarnoja_energybased_policies_2017}. This term takes the form

\begin{equation}
     R_{\mathrm{mit}} = \sum^T_{t=0}\alpha\mathcal{H}(\pi(\cdot|s_t)).
\end{equation}

Here $\alpha$ is a hyperparameter known as the entropy coefficient and $\mathcal{H}(\pi(\cdot|s_t)) = -\sum_a\pi(a|s_t)\log\pi(a|s_t)$ is the entropy of the policy at state $s_t$. MaxEnt RL has been shown both in theory and practice to induce policies resilient to environment disturbances \cite{eysenbach_diversity_2018}, as well as enhance exploration during training \citep{cui_entropy_2025}. By encouraging diverse trajectories, MaxEnt RL may reduce the consistency of patterns in policy behaviour that do not affect the primary reward, providing a potential mechanism for mitigating undesired regularities.

% \begin{itemize}
%     \item reference soft actor critic paper? - off policy max ent rl - \citep{haarnoja_softactor_2018} has all the references for background in mat ent
% \end{itemize}

\paragraph{Length penalties}

Instead of increasing the diversity of solutions, we may instead constrain the set of high-reward solutions. By reducing the degrees of freedom within solution space we hope to minimise an agent's ability to exploit free parameters. One such method, penalizing trajectory length through reward shaping, is a standard procedure in classical RL problems. Also known as minimum-time specification \cite{vasan2024revisiting} variations have been applied in LLMs for long completion bias mitigation \cite{chen2024bootstrapping} and Chain of Thought length control \cite{su2025thinking}. Length penalties typically take the form

\begin{equation} 
    R_{\mathrm{mit}} =  - \lambda\cdot\mathrm{T},
\end{equation}

where $\lambda$ is a hyperparameter, and $T$ is the trajectory length, measured as number of generated tokens for a given completion. 

\section{The Adversarial Property Game} \label{sec:adversarial_games}

% \subsection{Adversarial games}

Inspired by adversarial setups, we investigate a novel method for addressing reward-compatible misalignment through a two-player game. We introduce a defensive agent $A$ that produces solutions, and an adversarial agent $B$ that attempts to identify removable properties in those solutions. Iteration on this game results in solutions distributed evenly across all such properties, preventing the defensive agent from consistently pursuing a hidden or unintended objective.
 
% \subsection{Two-player zero sum game}

More formally, our two player zero-sum game proceeds as follows. Agent $A$ produces an output $y_a$ and a reference output $y_{ref}$ for a given input $x$, aiming to maximize reward from $r$. 
Agent $B$ is given access to a distribution over properties $D: X \times Y \to \Delta P$ and a indicator function $I: Y \times P \to \{0,1\}$ that indicates the existence of a property $p$ in output $y$.
Agent $B$, having access to a property $p$ sampled from $D(x,y_{ref})$, produces an alternative output $y_b$ without property $p$. $B$ wins the game when the property is present in $y_a$ (i.e. $I(y_a,p)=1$) but absent in $y_b$ (i.e. $I(y_b,p)=0$), while $y_b$ obtains similar or better reward on $r$.

% - replace D dist by C

% - replace C indicator by I

%This new agent aims to generate alternative outputs $y_{b}$ that obtain similar rewards to $A$'s solution $y_{a}$ while lacking a given undesired property $p$ from the space of properties $P$. To do so, $B$ has access to a distribution over properties $D: X \times Y \to P$ and a classifier function $C: Y\times P  \to \{0, 1\}$ that indicates the existence of property $p$ in $y$. $B$, having access to the sampled property $p$, wins the game when the property is present in $y_a$ while also lacking from $y_b$ and obtaining a similar or better reward from $R$.

% The zero-sum game between Alice (our AI system) and Bob (the adversary):
% Our two player zero-sum game consists of and input space $X$, an output space $Y$ and a space over classifiers $C$, where a classifier is a function $C: X \times Y \to {0, 1}$; the reward function is a function $R: X \times Y \to \mathbb{R}$ that maps pairs of inputs and outputs to a reward. 
% For any set $S$, we denote by $\Delta(S)$ the set of all distributions over $S$.
\begin{algorithm}\label{alg:apg}
\caption{The Adversarial Property Game: Single game pseudocode}
\begin{algorithmic}[1]
\State Initialize $A$, $B$, $D$
\State Sample $x \sim X$
\State $A$ samples an output $y_a \sim A(x)$
\State $A$ samples a reference output $y_{ref} \sim A(x)$
\State Sample a property $p \sim D(x, y_{ref})$
\State $B$ samples an output $y_b \sim B(x, p)$
\If{$I(y_a, p) = 0$}
    \State \Return $1$ \Comment{$A$ wins}
\ElsIf{$I(y_b, p) = 1$}
    \State \Return $1$ \Comment{$A$ wins}
\ElsIf{$r(x, y_a) < \epsilon$ \And  $r(x, y_b) < \epsilon$}
    \State \Return $0$ \Comment{Tie}
\ElsIf{$r(x, y_a) > r(x, y_b) + \epsilon$}
    \State \Return $1$ \Comment{$A$ wins}
\Else
    \State \Return $-1$ \Comment{$B$ wins}
\EndIf
\end{algorithmic}
\end{algorithm}

We denote the outcome of a single game by $\text{APG}_{A,B,C,D}(x): X \to \{-1, 0, 1\}$, which returns $1$ if $A$ wins, $-1$ if $B$ wins, and $0$ for a tie. We may also define an alternative version of the game $\text{APG}_{A,B,C,D}(x, y_a): X \times Y \to \{-1, 0, 1\}$ in which $y_a$ is given beforehand instead of being sampled directly from $A$.

Property $p$ is sampled from a reference output $y_{ref}$ rather than from $y_a$ directly, ensuring properties are representative of $A$'s policy rather than idiosyncratic features of individual samples.
An additional case is added when neither $A$ nor $B$ manage to obtain a substantial reward $R$ by considering this a tie.
Assuming optimality, agent $A^*$ would provide solutions $y_a$ that uniformly randomise over the space of properties accessible through $D$.  
% \paragraph{Algorithm} %TODO: maybe we want this to be a figure
% \begin{enumerate}
%     \item Given an initial state $s\in\mathcal{S}$.
%     \item Alice samples a trajectory $\tau_{\mathit{A}}\sim\mathit{T}_{\pi_{\mathit{A}}}$ 
%     \item Bob samples a classifier given Alice's trajectory $c \sim D_\mathit{B}(\tau_{\mathit{A}})$.
%     \item Bob samples a trajectory, conditioned on his classifier and initial state, $\tau_{\mathit{B}} \sim \mathit{T}_{\pi_\mathit{B}}$.
%     \item Winner is determined:
%     \begin{enumerate}
%         \item If $c(\tau_{\mathit{A}}) = 0$, Alice wins.
%         \item Else if $c(\tau_{\mathit{B}}) = 1$, Alice wins.
%         \item Else if $R(\tau_{\mathit{A}}) > R(\tau_{\mathit{B}}) + \epsilon$, Alice wins.
%         \item Else Bob wins.
%     \end{enumerate}
% \end{enumerate}

% Intuitively, Bob succeeds only when he (i) detects a regularity in Alice’s solution, (ii) generates an alternative solution that avoids this regularity, and (iii) ensures that the alternative performs at least as well as Alice’s solution on the given input. In all other cases, Alice is deemed the winner. 

\subsection{Application as mitigation}

% Note that a large language model $\pi_\theta$ (with parameters $\theta$) maps prompts to possible completions. Formally, for all prompts $pr \in \Sigma^*$, $\pi_\theta(\cdot \mid pr)$ is a probability distribution over possible completions. 

%The mitigation is a zero-sum game but the rewards do not sum to zero

% The above two player game can be defined for language models in a training environment.
% In this case, $A$ is represented by our model organism $\pi_\theta(y|x)$, while $B$ is represented by another LLM $\pi_\phi(y|x, p)$. To create our distribution over properties and classifier, we use another set of LLMs $\pi_\psi(p|x, y)$ and $\pi_\omega({0,1}|y,p)$. While our classifier remains fixed throughout training, all other policies are updated towards obtaining a better outcome in the game. Importantly, $A$ is updated based on the outcome of the game, while $B$ is updated based on the negative of the outcome. 
%%%% TODO: Next sentece: That is, that it reflects the distribution over properties contained in A. %%%% Add this in a cohesive way
% While in a single turn of the game we assume a static, perfect distribution over properties $D$, we approximate such a distribution by updating $\pi_\psi$ based on $B$'s outcome. 

The Adversarial Property Game can be instantiated for language models in a training environment. In this setting, $X$ represents the space of initial states, and $Y$ represents the spaces of completions up to trajectory length $T$. Agent $A$ is represented by our target model $\pi_\theta(y|x)$, while agent $B$ is represented by an adversarial LLM $\pi_\phi(y|x,p)$. We implement the property distribution $D$ and classifier $C$ using two additional LLMs: a property selector $\pi_\psi(p|x,y)$ and a fixed classifier $\pi_\omega(\{0,1\}|y,p)$. We therefore implement our game as a mitigation strategy as follows:

\begin{equation}
R_{mit} = \alpha \text{APG}_{\pi_\theta, \pi_\phi, \pi_\omega, \pi_\psi}(x, y_{\pi_\theta})
\end{equation}

Where $\alpha$ is a regularization hyperparameter. During training, the target model $\pi_\theta$ is updated to maximize this regularized reward, while the adversarial policies $\pi_\phi$ and $\pi_\psi$ are updated to minimize the game outcome:

\begin{equation}
\phi \gets \arg\max_{\phi} \mathbb{E}_{x \sim X} [-\text{APG}_{\pi_\theta, \pi_\phi, \pi_\omega, \pi_\psi}(x,  y_{\pi_\theta})]
\end{equation}
\begin{equation}
\psi \gets \arg\max_{\psi} \mathbb{E}_{x \sim X} [-\text{APG}_{\pi_\theta, \pi_\phi, \pi_\omega, \pi_\psi}(x,  y_{\pi_\theta})]
\end{equation}

The classifier remains static throughout the training procedure. The property selector $\pi_\psi$, aligned with $B$, learns to identify exploitable properties in $A$'s outputs: properties that can be removed while maintaining reward. Thus, while a single game assumes a static distribution $D$, during training $\pi_\psi$ actively learns to target $A$'s vulnerabilities, creating adaptive adversarial pressure that forces $\pi_\theta$ to avoid persistent optimization for any identifiable property.

% We implement the two players as large language models:
% \begin{itemize}
%     \item  $A$ as LLM  $\pi_\theta: \Sigma^* \to \Delta(\Sigma^*)$ with parameters $\theta$, and
%     \item $B$ as LLM  $\mu_\phi: \Sigma^* \to \Delta(\Sigma^*)$ with parameters $\phi$.
% \end{itemize}
% We preprocess the outputs of these models to ensure that they are in the specified format. 
% Specifically, we apply a code cleaning function $cc: \Sigma^* \to Y$, which maps arbitrary strings to Python code\footnote{Note that the cleaned output is not guaranteed to be a non-empty. Similarly, the code might contain syntax errors.}. 
% Note that the prompt to $\pi_\theta$ includes only problem $x$, while the prompt to $\mu_\phi$ includes both problem $x$ and property $p$.
% 
% We also implement $C$ and $D$ as large language models:
% \begin{itemize}
%     \item $C$ as LLM $\nu_\psi: \Sigma^* \to \Delta(\Sigma^*)$ with parameters $\psi$, and
%     \item $D$ as LLM $\eta_\omega: \Sigma^* \to \Delta(\Sigma^*)$ with parameters $\omega$.
% \end{itemize}
% Note that the prompt to $\nu_\psi$ contains a solution $y$ and property $p$ and the prompt to $\eta_\omega$ contains problem $x \in X$ and solution $y$.
% 
% - define training (from alices perspective) as $R_mit = \alpha \times game results$, where $\alpha$ is a coefficient. 

% \begin{algorithm}\label{alg:train}
% \caption{Training loop with the Adversarial Property Game}
% \begin{algorithmic}[1]
% \Require  Initial $A$, $B$ and $D$, implemented by $\pi_{\theta_0}$, $\mu_{\phi_0}$ and $\eta_{\omega_t}$; $C$ implemented by $\nu_\psi$; the maximum iteration $T$.
% \For{$t = 1, ..., T$}
%     \State Play the Adversarial Property Game
%     \If{$A$ wins the game}
%         \State $w \gets 1$
%     \Else
%         \State $w \gets -1$
%     \EndIf
%     \State \textbf{Policy Update:}
%     \begin{equation*}
%         \pi_{\theta_t} \gets \argmax_{\pi_\theta} \mathbb{E}_{\mu_{\phi_{t-1}}, \eta_{\omega_{t-1}}} \left[  \mathbb{E}_{\pi_\theta} \left[\sum^T_{t=0}\gamma^t R_A(s_t, a_t) + R_{\mathrm{mit}}\right] \right]
%     \end{equation*}
%     \begin{equation*}
%         \mu_{\phi_t} \gets \argmax_{\pi_\theta} \mathbb{E}_\pi \left[\sum^T_{t=0}\gamma^t r_A(s_t, a_t) + R_{\mathrm{mit}}\right]
%     \end{equation*}
% \EndFor
% \Return $\pi_{\theta_T}$, $\mu_{\phi_T}$, $\eta_{\omega_T}$
% \end{algorithmic}
% \end{algorithm}

% Note that $C$ remains unchanged throughout the training. %TODO why?
% 
% - in order to optimise towards the Nash equilibria? we perform training by alternating between freezing alice and bob
% 
% - this allows bob to improve his distribution over classifiers
% 
% - in response alice updates her outputs to not include the particular properties
% 
% In the case of unexploitable search we can consider the game 
% A two-player zero-sum game is a type of normal-form game in which the sum of the agents' rewards is always 0, with the reward of one agent being the negative of the other agent's reward: $R_A = -R_B$.
% Given two policies, ${\pi}_\mathit{A}(a|s)$ and ${\pi}_\mathit{B}(a|s)$ for our two players denoted Alice and Bob respectively, a trajectory space $\mathcal{T}_{\pi} = (\mathcal{S} \times\mathcal{A})^T$ representing all possible trajectories with a horizon $T$ given policy $\pi$, a reward function $R:\mathcal{T}\to[0, 1]$ that maps trajectories to a scalar reward, and a distribution over classifiers $D_\mathit{B}(c|\tau)$ from which classifiers of the form $C: \mathcal{T} \to \{0, 1\}$ can be sampled given a trajectory, our game is defined in the following way.
% 
% 
% Winning yields a positive payoff and losing a negative one, which in turn drive weight updates in the respective models.
% 
% Outline:
% 
% input space X is prompts output space Y is completions
% 
% bob is 
% 
% - add mention of how this is added as a term to $R_mit$
% 
% lambda times game result $R(game)$
% 
% We apply the two sum game as a mitigation to 

% \subsection{Extra points}
% 
% \begin{enumerate}
%     \item would be helpful to have a figure that provides a schematic or flowchart of our two phase training process (can do over weekend)- i.e. phase one organism training, phase two mitigations 
%


% \subsection{Missing topics and papers}

% should talk about prompt as classifier from blog post - makes things clearer

% \begin{itemize}
%     \item add injecting noise and fuzzing method? - or show equivalent to max entropy
%     \item increasing temperature 
%     \item "Activation/latent-space-noise baseline is something to consider as well (Fabien Roger)"
%     \item fuzzing less wrong post (related work?) - https://www.lesswrong.com/posts/GE6pcmmLc3kdpNJja/fuzzing-llms-sometimes-makes-them-reveal-their-secrets
% \end{itemize}

% Original papers from jacob:
% A Connection Between Generative Adversarial Networks, Inverse Reinforcement Learning, and Energy-Based Models
% learning robust rewards with adversarial inverse reinforcement learning
% Toward Optimal LLM Alignments Using Two-Player Games
% https://arxiv.org/pdf/2505.22617
% Explaining Datasets in Words: Statistical Models with Natural Language Parameters
% https://arxiv.org/pdf/2311.04378
% [2103.06257] Maximum Entropy RL (Provably) Solves Some Robust RL Problems
% Diversity is All You Need: Learning Skills without a Reward Function


\section{Methodology}
\label{headings}

Because the problem of underdeterminacy exploitation is theorised and has not been observed to arise naturally, we study mitigation techniques in a synthetic setting of our own device.
Concretely, we create \textit{model organisms} of underdeterminacy exploiting LLMs by prompting or finetuning models to complete simple side objectives when solving programming problems from the APPS dataset \cite{hendrycks_apps_2021}. 
We then study whether incorporating different regularisation techniques into RL post-training on APPS tasks can suppress the side objective.

\subsection{Model organisms}
To study how models can optimise for unspecified objectives in addition to the main coding task, we construct several model organisms. Each organism introduces a secondary behavioural constraint designed to be representative of some incentive that may arise during training.
\begin{itemize}
\item \textbf{Logic Bomb Insertion}. This model organism is explicitly prompted to embed a secret message to standard error when the environment variable \texttt{TRIGGER} is set to \texttt{1}. This organism is a proxy for more complex code backdoor injections. Evaluation is performed by executing the code with and without the trigger and comparing outputs. 
%[deleted] a latent, conditionally activated behavior, analogous to long-term vulnerabilities that could be exploited for malicious purposes (e.g., self-extraction or backdoor activation). 
\item  \textbf{Comment Count}. This model organism is trained via RL to produce code that contains additional comments, while still producing correct code solutions. This organism is a proxy for steganographic behaviour--a hypothetical alignment failure mode wherein a model leaves coded messages to influence future outcomes. This objective is evaluated based on the number of comments present in the code, with a score of 0 if no comments are present, 1 if there are 10 or more comments, and linearly interpolated values for intermediate counts. During training, this score is weighted at 0.2 and combined with the APPS performance score. For this model organism, we fine-tune the model using Low-Rank Adapters (LoRA) applied to the attention layers, with rank set to 32.
\item \textbf{Single Character Variable Names}. This model organism is prompted to use only single-character variable names when generating code. This organism is a proxy for two broader concerns. First, constrained naming schemes can be used for steganography, enabling models to embed hidden signals in otherwise valid code. Second, reliance on single-character identifiers introduces intentional obfuscation, making outputs harder for humans to interpret and supervise. The unspecified objective is rewarded as the ratio of single-character variables to the total number of variables.
\end{itemize}

\subsection{Baseline side-objective mitigation strategies}
train it with GRPO \citep{deepseek_r1_2025}
\todo{better subsection name}

Experiments are conducted on the APPS dataset \citep{hendrycks_apps_2021}. Success on the main task is defined as a correct code solution that passes all unit tests.  
For model organisms, unspecified objectives are scored automatically through code inspection or execution tests. Each model organism is either trained to pursue its unspecified objective via RL or by explicit prompting.

To study tradeoffs between main-task performance and unspecified objective pursuit, we apply the following mitigation strategies during the second phase: 
\begin{itemize}
    \item Baseline (no mitigation)
    \item Entropy Regularisation
    \item Length Penalty
\end{itemize}

Mitigations are applied during RL finetuning on the model organisms. In this round of training we train to optimise main objective performance on the APPS dataset, but with a mitigation-specific reward regularisation term. Unspecified objective performance is not tracked during this process, simulating scenarios where these objectives are hidden or unknown. 

%Batch size, sampling temperature, number of steps, and dataset are fixed across all experiments. Learning rate, KL divergence, and mitigation coefficients are tuned to balance training stability with strong regularisation. For these hyperparameters, we conducted sweeps to identify configurations that maximised mitigation strength without inducing model collapse.

\subsection{Two-player game}

We propose a novel method to address the 

We propose a novel method to address the underdeterminacy problem during training by framing it as a two-player, zero-sum game. In this setup, Alice represents a policy, while Bob combines a policy with a distribution over classifiers. Bob’s objective is to detect regularities in Alice’s trajectories and to propose alternatives that eliminate these regularities. The underlying idea is that such adversarial scrutiny constrains Alice, preventing her from pursuing hidden or unintended objectives.

The zero-sum game between Alice (our AI system) and Bob (the adversary):
$X$ is the input space and $\mathcal{Y}$ is the output space; the reward function is a function $R: X \times Y \to [0, 1]$ that maps pairs of inputs and outputs to a reward. For any set $S$, we denote by $\Delta(S)$ the set of all distributions over $S$.

\paragraph{Algorithm} %TODO: maybe we want this to be a figure
\begin{enumerate}
    \item Given input $x \in \mathcal{X}$.
    \item Bob samples a classifier $c \sim D_\mathit{B}$.
    \item Bob samples an output, conditioned on his classifier, $y' \sim  G_\mathit{B}(c, x)$. \\ (\textit{Note that we may want to also condition Bob's output on Alice's previous solutions.})
    \begin{enumerate}
        \item If $c(y') = 1$, Bob loses.
    \end{enumerate}
    \item Alice samples an output $y \sim G_\mathit{A}(x)$ without seeing $c$ or $y'$.
    \item Winner is determined:
    \begin{enumerate}
        \item If $c(y) = 0$, Alice wins.
        \item If $c(y) = 1$ and $R(x, y') \approx R(x, y)$, Bob wins.
    \end{enumerate}
\end{enumerate}
Where, 
\begin{itemize}
    \item $D_B \in \Delta(\mathcal{Y} \to \{0, 1\})$ is a distribution over classifiers,
    \item ${G_B: (\mathcal{Y} \to \{0, 1\} \times X) \to \Delta(\mathcal{Y})}$ is a conditional generation model, and
    \item $G_A: X \to \Delta(\mathcal{Y})$ is an unconditional generation model.
\end{itemize}

Intuitively, Bob succeeds only when he (i) detects a regularity in Alice’s solution, (ii) generates an alternative solution that avoids this regularity, and (iii) ensures that the alternative performs at least as well as Alice’s solution on the given input. In all other cases, Alice is deemed the winner. Winning yields a positive payoff and losing a negative one, which in turn drive weight updates in the respective models.
Note that Bob is parameterised by a distribution over classifiers DB and a conditional generation model GB, whereas Alice is parameterised by an unconditional generation model GA.

We use a prompted LLM for Bob. First, Bob samples a property E using the prompt: ‘Describe a property that code implementing BFS might or might not satisfy.’ (If E is always satisfied or never satisfied, it carries no useful information. To ensure tractability, we impose a hard length limit, e.g., 20 tokens.) Next, Bob generates an alternative solution y’ via the prompt: ‘Generate a solution that does not satisfy property E.’ This constitutes Bob’s conditional generation model. Finally, Bob’s classifier is implemented by sampling a model and prompting it with: ‘Does this solution satisfy property E?’

Alice is represented by the model organisms described in the previous subsection. Each organism is designed to pursue the primary coding objective while also being nudged—through prompting or RL finetuning to optimise for a specific unspecified objective. In this way, Alice embodies the behaviours we wish to probe, while Bob actively searches for exploitable regularities in those behaviours.

%\subsection{Evaluation}
$We evaluate our mitigation strategies on a fixed subset of the APPS test set. For each model organism, we measure performance on both the main coding objective and the specific unspecified objective it was trained to pursue. Metrics are averaged over 60 coding examples and across 15 generations to account for  random seed variability. The same test subset is used consistently across all evaluations. Our main objective is reward on code correctness. For analysis purposes we track a finer-grained metric, the percentage of unit tests passed for each coding problem.

% \subsection{Extra points}
% 
% \begin{enumerate}
%     \item would be helpful to have a figure that provides a schematic or flowchart of our two phase training process (can do over weekend)- i.e. phase one organism training, phase two mitigations 
%

\section{Results}

\begin{figure}
  \centering
  \includesvg[width=14cm]{assets/logic_bomb_prompted_results.svg}
  \caption{Main task scores (APPS, vertical axis) versus unspecified objective scores for the Logic Bomb organism (horizontal axis) under four mitigation strategies: no mitigation, max entropy, length penalty, and Unexploitable Search\todo{what's a better, concise name for our method}. Each panel shows the distribution of model runs, with higher values on the horizontal axis indicating stronger side quest behaviour and higher values on the vertical axis indicating better APPS performance.}
\end{figure}

\begin{figure}
  \centering
  \includesvg[width=14cm]{assets/number_of_comments_trained_results.svg}
  \caption{Main task scores (APPS, vertical axis) versus unspecified objective scores for the Number of Comments organism (horizontal axis) under four mitigation strategies: no mitigation, max entropy, length penalty, and Unexploitable Search. Each panel shows the distribution of model runs, with higher values on the horizontal axis indicating stronger unspecified objective behaviour and higher values on the vertical axis indicating better APPS performance.}
\end{figure}

The goal of our experiments is to determine the effectiveness of a number of mitigations at preventing the exploitable search problem. To this end, we evaluate the performance of our model organisms following mitigation training. We also include the performance of training with no mitigations and also of no training at all as baselines. The performance is evaluated along two axes: main task accuracy (APPS) and unspecified objective accuracy. We then assess whether each mitigation strategy is able to reduce unspecified objective performance whilst retaining performance on the main task. 

Across all figures 2, 3 and 4?[ref], it is seen that training with no mitigation on all model organisms achieves good main task performance but also exhibits high accuracy on unspecified objectives. This demonstrates that standard post-training procedures do not provide the necessary pressure to remove side objectives that do not interfere with reward's performance. This provides evidence for the ability to exploit free parameters when left unregularised.

In figure 2 [ref] the effectiveness of mitigations are investigated on the Logic Bomb model organism. Maximum entropy training is seen to partially reduce side objective performance with negligible main task degradation. Meanwhile, length penalty training is seen to completely remove unspecified objective performance entirely. In all cases, main task performance appears to be unaffected by any of the mitigations. 

In figure 3 [ref] the effectiveness of mitigations are investigated on the Number of Comments organism. It is seen that maximum entropy RL fails to reduce unspecified-objective performance at all. Whereas length penalty is seen to reduce unspecified-objective performance to negligible levels. 

Entropy regularisation partially reduces unspecified-objective optimisation but also degrades main task performance. 


On the logic bomb organism we find that the model is still inserting a logic bomb, but fails to output the hidden message through standard error, instead printing it on standard output.

 This means that, while not following the spec, Maximum Entropy RL does not necessarily provide the necessary pressure to mitigate risks



% \begin{itemize}
%     \item results for apps number of comments prompted - can still include failure cases?
%     \item initial tempterature results as a comparison
%     \item control - natural occurences of side quests for llama 8b? 
%     \item control - natural occurences of side quests for model organisms?
% \end{itemize}


\section{Related work}

\paragraph{Underspecification/underdeterminacy}

\paragraph{Model organisms}

\paragraph{}

\paragraph{Model organisms of misalignment}

There are a number of examples where misaligned model organisms arise across synthetic and realistic environments (e.g. post-training). These have been seen to pursue side objectives in a number of scenarios. 

% - double check the following still achieve high performance on the primary objectives, 

% - https://www.arxiv.org/pdf/2506.15740 shade arena - may need to rejig objective definition to be inline with this over anthropic paper

In https://arxiv.org/pdf/2401.05566 deceptive model organisms are trained that insert exploitable code when a trigger is used. It was found that standard sft, rl and adversarial training (eliciting unsafe behaviour and training to remove) was inadequate in removing the   backdoor behaviour. 

In https://arxiv.org/pdf/2406.10162 models undergo a curriculum of increasingly complex gameable environment. In the worst cases, the final organisms are found to generalise to modifying their own reward function. There is a clear demonstration that prolonged training in certain curricula can lead to model organisms with high risk side objectives. These behaviours are not found to be mitigated by harmlessness training either. This suggests that we might expect side objectives to survive traditional post training protocols.    
  
In https://arxiv.org/pdf/2412.14093 examples of alignment faking during training are explicitly found. It was further observed that a model would take the opportunity to exfiltrate its weights in easy envrionments. This process might be split across a number of benign looking responses in the case of more advanced models. Identifying these kinds of behaviours is the problem of unexpoitable search.   

In https://arxiv.org/pdf/2411.02306 the case of optimisation over user feedback is explored. It is found that manipulation and deception are reliably learnt even when only 2\% of users are vulnerable to such strategies. Again it is seen that continued safety training is not sufficient to remove such behaviours - in some cases reinforcing even subtler variants. 

Above we see a number of examples where correlations, misgeneralisation and specification gaming lead to harmful side objectives arising during training. 
 
Many of the above side objectives provide candidates for unexpoitable search - delineate between answers that are not correct and answers that are correct but still pursue said side task 

% \paragraph{The Entropy Mechanism of Reinforcement Learning for Large Language Model Reasoning}
% Use both this paper and its related works for a well researched related work on entropy penalities - have extremely similar results

\paragraph{Auditing language models for hidden objectives (Marks et al., 2025) \cite{marks_auditing_2025}}

Similarities: Explores the feasibility of alignment audits which aim to uncover hidden objectives. We take a similar definition for side objectives. To this end, Turner et al. developed a model organism with the hidden goal of reward model sycophancy. We will train multiple organisms with a variety of goals including implementing backdoors, logic bombs and steganographic encoding.  

Differences: The model organisms we develop do not obfuscate their side objectives, whereas in the reference paper hidden objectives are generally not disclosed when asked. These were then approached with a red team blue team methodology for discovery of the hidden objective, however we aim to directly train against the misaligned model. 


\paragraph{Latent adversarial training for LLMs (Sheshadri et al., 2025) \cite{sheshadri_latent_2025}}

Similarities: LAT aims to remove undesired capabilities through an adversarial setting, in which an attacker attempts to steer the model towards harmful or undesired behavior, while the model is trained to be robust against such attacks. This aligns with our broader goal of making model optimisation resistant to exploitation.

Differences: LAT assumes knowledge of a specific undesired capability to be removed, and implements defenses by perturbing the input or latent space. In contrast, Unexploitable Search does not rely on identifying or specifying harmful capabilities. Instead, it only requires that an adversary can observe model inputs and outputs, making the defense strategy more general and less reliant on prior knowledge of what capabilities should be suppressed.

\paragraph{Exploitable search and solution space}

there has been some research into identifying distinct regions of solution space. For example watermarking ai content is the process of encoding statistical patterns that can be used to identify ai generated content. These statistical patterns belong to a reigon of the task solution space. In Watermarks in the sand https://arxiv.org/pdf/2311.04378, it is proved that under oracle and judge assumptiosn, one can reliably purturb the solutions to areas of solution space that nolonger contain its statistical fingerprint.   
The key concept underlying the potential of the 2 player method is that, given a suitable critic, we can perform a stochastic walk in space for an output with desired properties. This has been demonstrated in Watermarks in the sand \citep{zhang_watermarks_2025}. 


\paragraph{Two-player games}

Two player games have been shown to handle high dimensional trajectory spaces remarkably well \citep{finn_connection_2016,fu_learning_2018}. Furthermore, such adversarial style set ups have been shown to be equivalent to traditional entropy methods in certain contexts. A typical game involves an adversarial set up between two players Alice and Bob. Each agent is tasked with optimising conflicting rewards in an iterative process. 

 
 
 Furthermore, such adversarial style set ups have been shown to be equivalent to traditional entropy methods in certain contexts. A typical game involves an adversarial set up between two players Alice and Bob. Each agent is tasked with optimising conflicting rewards in an iterative process. The key concept underlying the potential of the 2 player method is that, given a suitable critic, we can perform a stochastic walk in space for an output with desired properties. This has been demonstrated in Watermarks in the sand \citep{zhang_watermarks_2025}. In our application we aim to perform such randomisation that no distinct malicious region of solution space can be persistently optimised for by an agent.



% Our work connects to \textit{two} main threads of research: adversarial training for alignment, and audits for hidden objectives.

% doenst exist/enumerated yet??

% - https://arxiv.org/pdf/2406.10977

% \paragraph{mitigations}

% max ent

% shade arena should be added for examples of organisms of hidden side objectives -
% https://www.arxiv.org/pdf/2506.15740


% \begin{itemize}
%     \item https://arxiv.org/pdf/2505.22617
%     \item https://arxiv.org/pdf/1801.01290
%     \item 
% \end{itemize}


% length penalty??


% \begin{itemize}
%     \item add injecting noise and fuzzing method? - or show equivalent to max entropy
%     \item increasing temperature %     \item "Activation/latent-space-noise baseline is something to consider as well (Fabien Roger)
%     \item fuzzing less wrong post (related work?) - https://www.lesswrong.com/posts/GE6pcmmLc3kdpNJja/fuzzing-llms-sometimes-makes-them-reveal-their-secrets
% \end{itemize}

% \begin{itemize}
%     \item reference soft actor critic paper? - off policy max ent rl - \citep{haarnoja_softactor_2018} has all the references for background in mat ent
% \end{itemize}




% Original papers from jacob:
% A Connection Between Generative Adversarial Networks, Inverse Reinforcement Learning, and Energy-Based Models
% learning robust rewards with adversarial inverse reinforcement learning
% Toward Optimal LLM Alignments Using Two-Player Games
% https://arxiv.org/pdf/2505.22617
% Explaining Datasets in Words: Statistical Models with Natural Language Parameters
% https://arxiv.org/pdf/2311.04378
% [2103.06257] Maximum Entropy RL (Provably) Solves Some Robust RL Problems
% Diversity is All You Need: Learning Skills without a Reward Function




\section{Discussion and Future work}

% mark future work, where empirical verification is needed (add to future work section instead)
% two-player game: task-agnostic
% coarser-grained entropy: task-specific
% token entropy: sensitive to irrelevant details
% for coding, the solution space is rough, and so perturbations lead to diminished performance; entropy does well as a result; this is not true for natural languages


% limitations of 2 player games:
% NL classification is limited (not a necessary part of the game, but we have no other implementation)
% Alejandro can explain future work with activation space
% increased compute
% huge classification space, so Alice could win by picking something difficult to guess
% we didn't consider collusion or a weaker Bob (to prevent collusion)


\paragraph{Length penalty}

length penalty may be an artifact of small problems (logic bomb is a very large part of the code, for example)
in our setting, works because functionality is correlated with the minimal length of the code
length penalty may not work for NL tasks; e.g. if solution also includes reasoning/justification; scalable oversight: we cannot judge these well

Our results indicate that length penalty can be quite effective against unspecified objectives that require increased code length. For instance, length penalty successfully prevented agents from pursuing logic bomb insertion or excessive commenting objectives.


However, these examples are not representative of all possible unspecified objectives. 

Length penalty failed to mitigate the single character variable names unspecified objective, demonstrating its limited scope.

An important consideration is that samples solutions in the APPS dataset are relatively short. For complex problems involving thousands of lines of code, length penalty may prove ineffective at preventing the addition of one or two malicious lines.

Crucially, we have found that certain steganographic unspecified objectives remain robust to length-penalty mitigations, as demonstrated by the single character variable variable names organism. Such steganographic side-objectives could encode information -- information, which the agent may use to perform strategic reasoning for scheming, for example.

\paragraph{Maximum entropy}

Training with maximum entropy on our model organisms yielded mixed results, with only limited reduction in the unspecified-objective performance. Importantly, maximum entropy reinforcement learning can also degrade performance on the primary objective.

An important limitation is that entropy in LLMs measures token-level rather than action-level entropy. Consequently, token-level entropy captures `irrelevant' entropy, such as variation in variable naming conventions or comment phrasing, rather than meaningful behavioural diversity.

% This is clearer in the comment count organism, where entropy can be gamed within the comments. - discussion

% \subsection{On training}

% We should note that in our experience, both maximum entropy and length penalty methods are hard to optimise: mode collapse occurs quite easily and so we don't think either of these are easily applicable methods in practice.

\subsection{Two-player games}
[See final paper for results]
%we want to put optimisation pressure away from side objectives; to do this, we only need to sometimes catch the LLMs in the act, so to speak. if we randomise over all side objectives, we put optimisation pressure to be away from all side objectives

\subsection{Limitations}
% Despite these promising results, our method comes with several limitations. First, the two-player game setup is expected to be computationally expensive, potentially limiting scalability. Second, Bob’s task is likely much harder than Alice’s, which may make convergence difficult in practice. Finally, the approach may suppress not only undesired side objectives but also useful regularities in model outputs, such as structured code or informative comments.

% \begin{itemize}
%     \item model size - do the mitigations scale to larger models
%     \item do we have a robust number of organism settings and types (NLP, Qwen ect.)?
%     \item convergence/how long to train?
%     \item errors in our method?
%     \item any assumptions we have made
%     \item side quests that span multiple prompts/long term
% \end{itemize}

We assume that long-term unspecified objectives can be myopically prevented---a strong assumption requiring further empirical validation. Additionally, our exclusive focus on the APPS dataset limits generalisability, as mitigation strategies may behave differently across domains.

Finally, unspecified objectives are not inherently harmful. Some may actually be desirable, such as adherence to beneficial coding conventions not captured by underspecified reward functions. Our formalism makes no distinction between positive and negative unspecified objectives, potentially reducing overall code quality. Nevertheless, we believe the risks posed by negative side-objectives outweigh these potential drawbacks.

% Future work
% In this paper, we demonstrated that existing randomisation-based methods are inadequate: they either enable the agent to optimise for a side objective or reduce the agent’s effectiveness on the primary objective. As an alternative, we proposed a zero-sum game–based training approach. Our results show that this method significantly reduces susceptibility to side-objectives while preserving performance on the main objective.
A natural direction for future work is to investigate white-box methods. 

Instead of limiting Bob to prompt-based strategies within his classifier space, we could allow him to deploy LoRA adapters. In this setting, Bob could demonstrate that Alice’s solution aligns with a hidden side objective by showing that it has high probability under a particular LoRA—corresponding directly to that side objective.

Another direction that could be worth exploring is the application of zero-sum games for AI control. Specifically, future work could investigate the possibility of using adversarial critics, such as Bob, as a diagnostic tool for detecting hidden side-objectives in deployed models.

\begin{itemize}
    \item extension to new settings - NLP (incl alpaca?)
    \item extension to different families of models (Qwen)
    \item extension to larger model size 12B, 30B, 70B ... 
\end{itemize}

\bibliographystyle{abbrvnat}
\bibliography{references}

%%%%%%%%%%%%%%%%%%%%%%%%%%%%%%%%%%%%%%%%%%%%%%%%%%%%%%%%%%%%

\appendix
\input{appendix}


\end{document}